% Metódy inžinierskej práce

\documentclass[10pt,twoside,slovak,a4paper]{article}

\usepackage[slovak]{babel}
%\usepackage[T1]{fontenc}
\usepackage[IL2]{fontenc} % lepšia sadzba písmena Ľ než v T1
\usepackage[utf8]{inputenc}
\usepackage{graphicx}
\usepackage{url} % príkaz \url na formátovanie URL
\usepackage{hyperref} % odkazy v texte budú aktívne (pri niektorých triedach dokumentov spôsobuje posun textu)

\usepackage{cite}
%\usepackage{times}

%\pagestyle{headings}

\title{Softvéry na detekciu potencionálnych rizík 
implementované v automobiloch\thanks{Semestrálny projekt v predmete Metódy inžinierskej práce, ak. rok 2021/22, vedenie: PhD. Ing. Fedor Lehocki  }} % meno a priezvisko vyučujúceho na cvičeniach

\author{Michal Januška\\[2pt]
	{\small Slovenská technická univerzita v Bratislave}\\
	{\small Fakulta informatiky a informačných technológií}\\
	{\small \texttt{xjanuskam@stuba.sk}}
	}

\date{\small 5.10. 2021} % upravte



\begin{document}

\maketitle

\begin{abstract}
Žijeme v dobe, kde už je vlastníctvo automobilu komoditou. Tieto stroje sú nám veľkou výpomocou či už pri výkone práce alebo pri voľnočasových aktivitách. Aj keď sa na prvý pohľad môže zdať, že automobily so sebou prinášajú samé výhody, nastávajú aj také situácie kedy sa toto vozidlo môže vymknúť kontrole vodiča a môže dôjsť  k veľmi nebezpečným situáciam alebo až k smrteľným nehodám. V takýchto situáciach nastupujú rôzne zabudované softvéry, ktoré majú za úlohu včas detekovať hroziace riziko a vykonať všetko pre jeho zabránenie s dôrazom na čo najmenšie ohrozenie posádky automobilu ale aj vonkajšieho prostredia. Tieto softvéry fungujú na báze umelej inteligencie a machine learningu. V tejto práci sa venujem predstaveniu týchto softvérov z rôznych aspektov a ich funkcionalite ktorá spočíva v modelovaní postupov riešení pri rizikových situáciach.
\end{abstract}



\section{Úvod}\label{uvod}

Zameranie tejto práce je pomerne široké, preto bude pozostávať s viacerych sekcií. Hneď v sekcii~\ref{problematika} si predstavíme riziká ktoré môžu nastať pri vedení automobilu alebo aj pri automobile ktorý nie je práve v pohybe. V nasledujúcej sekcii~\ref{predstavenie} si predstavíme softvéry a pomocné komponenty ktoré majú za úlohu sa s týmito situáciami vysporiadať a urobíme ich podrobný rozbor. Veľa z týchto komponentov ešte momentalne nie je implementovaných v bežných automobiloch ale vývojári sa ich snažia dokončiť čo najrýchlejšie a následne vyslať na trh. Budeme sa zaoberať vytváraním modelov daných situácií a ukážeme si ich priebeh od začiatku až do konca. V tejto práci nebudem poukazovať len na výhody týchto softvérov a zariadení, ale budem uvádzať aj nevýhody a ich potencionálne odstránenie. Rozbor jednotlivých softvérov a komponentov bude obohatený o rôzne grafy a vývojové diagramy, pre lepšie pochopenie danej problematiky. Celá práca bude zhrnutá v sekcii~\ref{zaver}, kde sa podelím o svoj osobný názor a urobím jej kompletný výstup.



\section{Problematika} \label{problematika}

Pokroky v automobilovom priemysle sú veľmi citeľné a s dostupnosťou moderných technológií napredujú bleskovou rýchlosťou. Ako príklad nám slúži autopilot, ktorý bol prvotne implementovaný v autách značky Tesla a momentálne už máme na trhu veľké množstvo automobilových firiem ktoré taktiež implementovali túto funkcionalitu do svojich áut. Autopilot ako taký, je pomerne nová funkcionalita a preto s jeho nástupom prišli aj situácie v ktorých nezareagoval adekvátne alebo aj vôbec a tým pádom spôsobil závažné nehody. Zoberme si ako príklad rok 2018\cite{main}, kedy autopilotom vedené vozidlo značky Uber zrazilo chodca a spôsobilo mu vážnejšie zranenia. Aj napriek rozvoju autopilota, sme sa ešte nedostali do takej fázy aby bol rozšírený do väčšiny áut, najmä kvôli jeho cenovej dostupnosti, a preto drvivá väčšina obyvateľstva riadi svoje auto manuálne. Tu zasa prichádzajú ďaľšie faktory ako napríklad fyzický a psychický stav vodiča a posádky auta, ktoré môžu náhle zmeniť pokojnú situáciu vedenia vozidla na kritickú. Ide napríklad o náhlu zmenu zdravotného stavu šoféra, kedy stráca kontrolu nad vozidlom a ohrozuje tým svoju posádku a taktiež aaj vonkajšie okolie. Aby sa stihlo včas zabrániť takýmto situáciam, firmy prichádzajú s rôznymi softvérmi a komponentami ktorých funkcia je detekovať, vyhodnotiť a vyriešiť danú rizikovú situáciu.



\section{Predstavenie softvérov a komponentov} \label{predstavenie}

V predchádzajúcej sekcii~\ref{problematika} sme boli oboznámení s rizikovými sutiáciami ktoré môžu nastať. V tejto sekcii si predstavíme rôzne softvéry a komponenty ktoré majú týmto situáciam zabrániť. Tieto komponenty sú strategicky osadené vo výhodných pozíciach automobilu, aby dokázali pracovať bez akéhokoľvek obmedzenia.

\subsection{Audi Fit Driver\cite{main}} \label{predstavenie:audi}

Audi Fit Driver je koncept od spoločnosti Audi, ktorý má za pomoci získaných informácií o fyzickom a psychickom stave vodiča ale taktiež aj stave vonkajšieho prostredia, ako napríklad stav počasia alebo hustota dopravy, upraviť prostredie automobilu tak, aby sa znížila pravdepodobnosť vzniku nejkého rizika. Ak je zdravotný stav vodiča natoľko zlý, že stráca kontrolu nad vozidlom, aktivujú sa softvéry ktoré dokážu zavolať vodičovi záchranku ale taktiež bezpečne dopraviť automobil na najbližšie stojisko. Za zber týchto informácií zodpovedajú napríklad hodinky alebo náramok, ktorý komunikuje s vonkajšími servermi ale taktiež aj so softvérmi zabudovanými priamo v automobile. Ak je napríklad zaznamenaná zvýšená hladina stresu, ktorá sa môže preukázať zvýšeným pulzom šoféra, Audi Fit Driver napríklad upraví hlasitosť a štýl hudby ktorú vodič počúva, alebo v prípade hustej dopravy ponúkne alternatívnu, menej stresujúcu trasu.


\subsection{Pasenger monitoring using AI radar} \label{predstavenie:radar}

\subsection{Hearth rate monitoring system} \label{predstavenie:hearth}

\paragraph{Veľmi dôležitá poznámka.}
Niekedy je potrebné nadpisom označiť odsek. Text pokračuje hneď za nadpisom.



\section{Modelovanie situácií} \label{modelovanie}




\section{Ešte dôležitejšia časť} \label{dolezitejsia}




\section{Záver} \label{zaver} % prípadne iný variant názvu



%\acknowledgement{Ak niekomu chcete poďakovať\ldots}


% týmto sa generuje zoznam literatúry z obsahu súboru literatura.bib podľa toho, na čo sa v článku odkazujete

\bibliographystyle{plain} % prípadne alpha, abbrv alebo hociktorý iný
\bibliography{bibliografia.bib}
\end{document}