% Metódy inžinierskej práce

\documentclass[10pt,twoside,slovak,a4paper]{article}

\usepackage[slovak]{babel}
%\usepackage[T1]{fontenc}
\usepackage[IL2]{fontenc} % lepšia sadzba písmena Ľ než v T1
\usepackage[utf8]{inputenc}
\usepackage{graphicx}
\usepackage{url} % príkaz \url na formátovanie URL
\usepackage{hyperref} % odkazy v texte budú aktívne (pri niektorých triedach dokumentov spôsobuje posun textu)

\usepackage{cite}
%\usepackage{times}

%\pagestyle{headings}

\title{Softvéry na detekciu potencionálnych rizík 
implementované v automobiloch\thanks{Semestrálny projekt v predmete Metódy inžinierskej práce, ak. rok 2021/22, vedenie: PhD. Ing. Fedor Lehocki  }} % meno a priezvisko vyučujúceho na cvičeniach

\author{Michal Januška\\[2pt]
	{\small Slovenská technická univerzita v Bratislave}\\
	{\small Fakulta informatiky a informačných technológií}\\
	{\small \texttt{xjanuskam@stuba.sk}}
	}

\date{\small 5.10. 2021} % upravte



\begin{document}

\maketitle

\begin{abstract}
Žijeme v dobe, kde už je vlastníctvo automobilu komoditou. Tieto stroje sú nám veľkou výpomocou či už pri výkone práce alebo pri voľnočasových aktivitách. Aj keď sa na prvý pohľad môže zdať, že automobily so sebou prinášajú samé výhody, nastávajú aj také situácie kedy sa toto vozidlo môže vymknúť kontrole vodiča a môže dôjsť až k smrteľným nehodám alebo situáciam. V takýchto situáciach nastupujú rôzne zabudované softvéry, ktoré majú za úlohu včas detekovať hroziace riziko a vykonať všetko pre jeho zabránenie s dôrazom na čo najmenšie ohrozenie posádky automobilu ale aj vonkajšieho prostredia. Tieto softvéry fungujú na báze umelej inteligencie a machine learningu. V tejto práci sa venujem predstaveniu týchto softvérov z rôznych aspektov a ich funkcionalite ktorá spočíva v modelovaní postupov riešení pri rizikových situáciach.
\end{abstract}



\section{Úvod}

Zameranie tejto práce je pomerne široké, preto bude pozostávať s viacerych častí. Hneď v sekcii~\ref{problematika} si predstavíme riziká ktoré môžu nastať pri vedení automobilu alebo aj pri automobile ktorý nie je práve v pohybe. V nasledujúcej sekcii~\ref{predstavenie} si predstavíme softvéry a pomocné komponenty ktoré majú za úlohu sa s týmito situáciami vysporiadať a urobíme ich podrobný rozbor. Veľa z týchto komponentov ešte momentalne nie je implementovaných v bežných automobiloch ale vývojári sa ich snažia dokončiť čo najrýchlejšie a následne vyslať na trh. Budeme sa zaoberať vytváraním modelov daných situácií a ukážeme si ich priebeh od začiatku až do konca. V tejto práci nebudem poukazovať len na výhody týchto softvérov a zariadení, ale budem uvádzať aj nevýhody a ich potencionálne odstránenie. Rozbor jednotlivých softvérov a komponentov bude obohatený o rôzne grafy a vývojové diagramy, pre lepšie pochopenie danej problematiky. Celá práca bude zhrnutá v sekcii Záver, kde sa podelím o svoj osobný názor a urobím jej kompletný výstup.



\section{Problematika} \label{problematika}

Z obr.~\ref{f:rozhod} je všetko jasné. 

\begin{figure*}[tbh]
\centering
%\includegraphics[scale=1.0]{diagram.pdf}
Aj text môže byť prezentovaný ako obrázok. Stane sa z neho označný plávajúci objekt. Po vytvorení diagramu zrušte znak \texttt{\%} pred príkazom \verb|\includegraphics| označte tento riadok ako komentár (tiež pomocou znaku \texttt{\%}).
\caption{Rozhodujúci argument.}
\label{f:rozhod}
\end{figure*}



\section{Predstavenie softvérov a komponentov} \label{predstavenie}

Základným problémom je teda\ldots{} Najprv sa pozrieme na nejaké vysvetlenie (časť~\ref{ina:nejake}), a potom na ešte nejaké (časť~\ref{ina:nejake}).\footnote{Niekedy môžete potrebovať aj poznámku pod čiarou.}

Môže sa zdať, že problém vlastne nejestvuje\cite{Coplien:MPD}, ale bolo dokázané, že to tak nie je~\cite{Czarnecki:Staged, Czarnecki:Progress}. Napriek tomu, aj dnes na webe narazíme na všelijaké pochybné názory\cite{PLP-Framework}. Dôležité veci možno \emph{zdôrazniť kurzívou}.


\subsection{Nejaké vysvetlenie} \label{ina:nejake}

Niekedy treba uviesť zoznam:

\begin{itemize}
\item jedna vec
\item druhá vec
	\begin{itemize}
	\item x
	\item y
	\end{itemize}
\end{itemize}

Ten istý zoznam, len číslovaný:

\begin{enumerate}
\item jedna vec
\item druhá vec
	\begin{enumerate}
	\item x
	\item y
	\end{enumerate}
\end{enumerate}


\subsection{Ešte nejaké vysvetlenie} \label{ina:este}

\paragraph{Veľmi dôležitá poznámka.}
Niekedy je potrebné nadpisom označiť odsek. Text pokračuje hneď za nadpisom.



\section{Dôležitá časť} \label{dolezita}




\section{Ešte dôležitejšia časť} \label{dolezitejsia}




\section{Záver} \label{zaver} % prípadne iný variant názvu



%\acknowledgement{Ak niekomu chcete poďakovať\ldots}


% týmto sa generuje zoznam literatúry z obsahu súboru literatura.bib podľa toho, na čo sa v článku odkazujete
\bibliography{bibliografia}
\bibliographystyle{plain} % prípadne alpha, abbrv alebo hociktorý iný
\end{document}